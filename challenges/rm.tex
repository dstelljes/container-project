\chapter*{Dies Irae}

\begin{quoting}
  Ah! what terror shall be shaping\newline
  When the Judge the truth's undraping---\newline
  Cats from every bag escaping!

  \begin{flushright}
    --- Ambrose Bierce, \textit{Shapes of Clay}
  \end{flushright}
\end{quoting}

\vspace{2em}

\section*{The challenge}

Destroy everything.

\section*{Getting started}

The only thing you'll need for this exercise is a machine with the LXD container engine on it. Installing LXD is sort of a pain (slightly less so if you're using Ubuntu or Gentoo), but you can try it out online:

\begin{enumerate}
  \item Point a web browser at \url{https://linuxcontainers.org/lxd/try-it/}.
  \item Acknowledge the terms.
  \item Observe that you're now root inside your very own LXD-enabled system.
\end{enumerate}

\section*{Creating containers}

Enter the following command into the shell:

\begin{verbatim}
    # lxc launch ubuntu adios-amigos
\end{verbatim}

A couple of things to note here: First, \texttt{lxc}---what's up with that? LXD is actually just a friendly interface to an older container management tool called LXC, and the name of the command reflects that. Pass judgment as you will.

\texttt{launch} does exactly what you'd think; it launches a container. In this case, you're launching a container based on the \texttt{ubuntu} (Ubuntu Linux) image, and you're naming it \texttt{adios-amigos}.

Notice that it takes awhile to pull down the \texttt{ubuntu} image. That only has to happen once. If you were to launch another container based on that image, it would happen almost instantly.

Check out a list of currently running containers and you'll see \texttt{adios-amigos} purring away:

\begin{verbatim}
    # lxc list
    +--------------+---------+
    |     NAME     |  STATE  |
    +--------------+---------+
    | adios-amigos | RUNNING |
    +--------------+---------+
\end{verbatim}

\section*{Wrecking containers}

To open a console inside of the container, use \texttt{lxc exec}:

\begin{verbatim}
    # lxc exec adios-amigos -- bash
\end{verbatim}

If that worked, you'll see that your prompt has changed to \texttt{root@adios-amigos}. You're in the container! Now, do some damage:

\begin{verbatim}
    # rm -rf --no-preserve-root /
\end{verbatim}

\texttt{rm} will spit out a pretty impressive list of stuff it can't delete, but everything else in the container will be destroyed. Try running \texttt{rm} again:

\begin{verbatim}
    # rm
    bash: /bin/rm: No such file or directory
\end{verbatim}

At this point, you don't have a whole lot of options other than leaving the container:

\begin{verbatim}
    # exit
\end{verbatim}

And, because everything is gone, you can't actually get back in:

\begin{verbatim}
    # lxc exec adios-amigos -- bash
\end{verbatim}

This time, your prompt doesn't change---\texttt{bash} was deleted.

\section*{Wrecking containers \textit{safely}}

LXD allows you to take \textit{snapshots} of running containers, so it's possible to do all sorts of destructive stuff without any long-term consequences. If you had taken a snapshot prior to destroying everything\ldots

\begin{verbatim}
    # lxc snapshot adios-amigos emergency-parachute
\end{verbatim}

\ldots you would have been able to restore the container to a working state:

\begin{verbatim}
    # lxc restore adios-amigos emergency-parachute
\end{verbatim}
