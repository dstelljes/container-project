\chapter*{Paula Deen's Thanksgiving}

\begin{quoting}
  Turducken is a dish consisting of a deboned chicken stuffed into a deboned duck, further stuffed into a deboned turkey. Outside of the United States and Canada, it is known as a three bird roast. Gooducken is a traditional English variant, replacing turkey with goose.

  \begin{flushright}
    --- Wikipedia, \textit{Turducken}
  \end{flushright}
\end{quoting}

\vspace{2em}

As we discussed briefly in the presentation, there are two main types of containers: \textit{system containers}, which are essentially lightweight virtual machines, and \textit{application containers}, which focus on packaging single applications.

LXD is a container engine that manages system containers. Working with an LXD container is a lot like working with a virtual machine (or, for that matter, a physical one) in that it acts like a completely isolated guest operating system.

Docker, by contrast, manages application containers that are designed to be stateless and easily recreated. If a Docker container gets sick, it can be swiftly Old Yeller'd (spoiler, sorry) and a new container can be created to take its place.

\section*{The challenge}

As it turns out, Docker will inside of LXD. There are some actual reasons why this kind of thing is useful, but for now we'll just observe that it's awesome and move on. In this exercise, you'll create a container Turducken, stuffing one container engine inside another. More butter, y'all!

\section*{Getting started}

For this exercise, you'll need a machine with LXD on it. We're running four cloud servers with LXD on them, so all you have to do is connect to one:

\begin{enumerate}
  \item In a terminal, type \texttt{ssh clontarf@lxd-1.stellj.es}. You can also use \texttt{lxd-2}, \texttt{lxd-3}, or \texttt{lxd-4}.
  \item We'll provide the password in the presentation.
  \item You have root privileges on the machine.
\end{enumerate}

Fun fact: The system you're in is a virtual machine, so you're already one level deep!

\section*{Duck, meet turkey}

Start out by launching a new container:

\begin{verbatim}
    $ lxc launch ubuntu turkey -c security.nesting=true -c security.privileged=true
\end{verbatim}

Barring any problems, you now have a container based on the \texttt{ubuntu} (Ubuntu Linux) image named \texttt{turkey}. As you may have guessed, \texttt{-c security.nesting=true} makes it possible for nesting (or Deen-ing, if you prefer) to happen.

Note also the \texttt{-c security.privileged=true}---that gives the container elevated privileges, which is a gross workaround for an issue that happens with nested LXDs. The neat thing is that the container is still unprivileged with respect to the host; it only has elevated privileges within \texttt{turkey}.

Use \texttt{lxc exec} to get into the container:

\begin{verbatim}
    $ lxc exec turkey -- bash
\end{verbatim}

If all goes well, you should see that your prompt is now \texttt{root@turkey}. The Ubuntu image ships with LXD installed, so you can simply initialize it and then repeat the process to add another layer:

\begin{verbatim}
    # lxd init --auto
    # lxc launch ubuntu duck -c security.nesting=true -c security.privileged=true
\end{verbatim}

\section*{We have to go deeper}

So we've seen LXD inside of LXD, but what about Docker inside of LXD? Use \texttt{apt} to install Docker and find out:

\begin{verbatim}
    # apt update
    # apt install -y docker.io
\end{verbatim}

And now, you should be able to interact with Docker:

\begin{verbatim}
    # docker info
    Containers: 0
     Running: 0
    ...
\end{verbatim}

Just for kicks, put a whale inside of the turkey:

\begin{verbatim}
    # docker run docker/whalesay cowsay sup
\end{verbatim}
