\chapter*{Paula Deen's Thanksgiving}

\begin{quoting}
  Turducken is a dish consisting of a deboned chicken stuffed into a deboned duck, further stuffed into a deboned turkey. Outside of the United States and Canada, it is known as a three bird roast. Gooducken is a traditional English variant, replacing turkey with goose.

  \begin{flushright}
    --- Wikipedia, \textit{Turducken}
  \end{flushright}
\end{quoting}

\vspace{2em}

As we discussed briefly in the presentation, there two main types of containers: \textit{system containers}, which act like complete isolated operating systems, and \textit{application containers}, which focus on packaging single applications.

LXD is a container engine that manages system containers. Working with an LXD container is a lot like working with a virtual machine (or, for that matter, a physical one) in that it acts like a completely isolated guest operating system.

Docker, by contrast, manages application containers that are designed to be stateless and easily recreated. If a Docker container gets sick, it can be swiftly Old Yeller'd (spoiler, sorry) and a new container can be created to take its place.

\section*{The challenge}

As it turns out, Docker will inside of LXD. There are some actual reasons why this kind of thing is useful, but for now we'll just observe that it's awesome and move on. In this exercise, you'll create a container Turducken, stuffing one container engine inside another. More butter, y'all!

\section*{Getting started}

The only thing you'll need for this exercise is a machine with LXD on it. Installing LXD is sort of a pain (slightly less so if you're using Ubuntu or Gentoo), but you can use LXD online:

\begin{enumerate}
  \item Point a web browser at \url{https://linuxcontainers.org/lxd/try-it/}.
  \item Acknowledge that you can't/won't do anything terrible.
  \item Observe that you're now root inside your very own LXD-enabled system.
\end{enumerate}

Fun fact: The ``try it online'' system is actually an LXD container running on top of a virtual machine. We're already a few levels deep!

\section*{Duck, meet turkey}

Start out by launching a new system container:

\begin{verbatim}
    # lxc launch ubuntu:16.04 first-container -p default -p docker
\end{verbatim}

There are a few things going on here. First, \texttt{lxc}---what's up with that? LXD is actually just a friendly interface to an older container management tool called LXC, and the name of the command reflects that. Pass judgment as you will.

\texttt{launch} does exactly what you'd think; it launches a container. In this case, you're launching a container based on the \texttt{ubuntu:16.04} (Ubuntu 16.04) image, and you're naming it \texttt{first-container}.

The \texttt{-p} options apply \textit{profiles} to the container. \texttt{-p default} sets up some networking configuration and \texttt{-p docker} makes it possible for nesting to happen.