\documentclass[xcolor=dvipsnames,aspectratio=1610]{beamer}
\usetheme[numbering=counter, progressbar=frametitle, sectionpage=none]{metropolis}
\title{Software Containers}
\date{December 5, 2016}
\author{Shawn Seymour, Dan Stelljes}
\setbeamercolor{title separator}{fg=Cyan!60}
\setbeamercolor{progress bar}{fg=NavyBlue}
\setbeamercolor{frametitle}{fg=Black,bg=Cyan!60}
\setbeamercolor{alerted text}{fg=BurntOrange}


\begin{document}
  \maketitle
  \begin{frame}{Virtual Machines vs. Containers}
      \alert{Virtual machines} act as hardware virtualization and run a full operating system as a guest.
      \begin{itemize}
          \item Can be several gigabytes in size and take awhile to start up.
      \end{itemize}
      \vspace{10px}

      \alert{Containers} act as operating system virtualization where they share the kernel of the host machine.
      \begin{itemize}
          \item Can be tens of megabytes and start up instantly.
      \end{itemize}

  \end{frame}

  \begin{frame}{Virtual Machines vs. Containers}
      \begin{figure}
        \includegraphics[scale=0.5]{container_vs_vm.jpg}
      \end{figure}
      Containers are used as a lightweight alternative to virtual machines and are very useful when deploying and running applications in clustered environments.

    \end{frame}


  \begin{frame}{Container Applications}
    So why containers?

    \begin{itemize}
        \item They can easily be spun up, worked with, destroyed, re-created, and built upon.
        \item Containers provide isolation; for example, in a web stack, the server and database can be hosted on the same machine but isolated in their own containers.
        \item It is very easy to scale applications over multiple containers as well as multiple hosts in a cluster setup.
    \end{itemize}
  \end{frame}

  \begin{frame}{Container Environments}
    There are many container environments to choose from:

    \begin{itemize}
        \item \alert{LXC} (\textbf{L}inu\textbf{X} \textbf{C}ontainers): Designed to run isolated Linux systems on a shared kernel. Made to contain an entire system. Been around over eight years.
        \item \alert{Docker}: A platform for developing, shipping, and deploying single applications. Currently the most common container environment.
        \item \alert{LXD}: An easier way to run LXC containers. Provides a simpler, better interface around the low-level tools provided by LXC.
        \item \alert{rkt}: The newest, cutting-edge container environment. Designed to address security issues with Docker and to standardize specifications for application containers.
    \end{itemize}
  \end{frame}

  \begin{frame}{Lab Suggestions}
    We did some cool things and proposed some ideas for the labs:

    \begin{itemize}
        \item \alert{Lab Services}:
        \begin{itemize}
            \item Trac / Puppet can be hosted in containers on a physical machine.
            \item LDAP can be hosted with container(s) with a possible load balancer.
        \end{itemize}

        \item \alert{Course Usages}:
        \begin{itemize}
            \item Computing Systems: Practicum (3403) could use LXD containers to blow up the world.
            \item Software Design (3601) could use Docker containers for deployment of their web server and database.
        \end{itemize}
    \end{itemize}
  \end{frame}

  \begin{frame}{Container Demos}
      We have a few demos of different software container applications to play around with: \newline
      \begin{itemize}
        \setlength\itemsep{1.6em}
        \item Dies Irae: Blow up the world challenge; example of LXD containers
        \item Paula Deen's Thanksgiving: Nesting Docker inside of LXD
        \item Down The Pipeline: Manipulating images by Docker and Rancher
      \end{itemize}

  \end{frame}

  \begin{frame}{Appendix B: Kyle Hakala or Seth Meyers?}

      \begin{figure}
      \centering
      \begin{minipage}{.5\textwidth}
        \centering
        \includegraphics[width=\linewidth]{kyle.png}
      \end{minipage}%
      \begin{minipage}{.5\textwidth}
        \centering
        \includegraphics[width=.8\linewidth]{seth.jpg}
      \end{minipage}
      \end{figure}

  \end{frame}

  \begin{frame}{Appendix B}
      \begin{figure}
        \includegraphics[scale=0.6]{cactus.png}
      \end{figure}
  \end{frame}

\end{document}
